\documentclass[14pt]{extreport}
\usepackage[russian]{babel}
% поля:
\usepackage[left=2.5cm, right=1.5cm, top=2.5cm, bottom=2.5cm]{geometry}
\title{ДЗ No1: Регулярные языки и конечные автоматы}
\author{Выполнил: студент группы\\ А-13б-19 Перепёлкин Дмитрий}
\date{Апрель 2022 год}
\usepackage{titlesec}
\usepackage{amsmath}
\usepackage[pdftex]{graphicx}
\DeclareGraphicsExtensions{.pdf,.png,.jpg}
\setlength{\parindent}{0pt}

\begin{document}

\maketitle

$\text{Мы будем часто обращаться к } \textbf{Лемме о разрастании} : $\\
$\text{Для бесконечного автоматного языка } L  \\
\text{над алфавитом } V \text{ существует такое натуральное число } n,  \\
\text{что для любого слова } \\
\alpha \in L  \text{ длины не меньше } n  \text{ найдутся слова } u,v,w\in V^* \text{ такие, что } \\
\alpha =uvw, |uv|\leq n, |v|\geq 1 \\
\text{и для всякого неотрицательного целого } i, uv^iw \text{ будет являться словом языка }L.$\\
$\left( \exists n \in N \right) \left(\forall\alpha \in L : |\alpha| \geq n \right) \left( \exists u,v,w \in V^* \right) : $\\
$[\alpha = uvw \wedge |uv| \leq n \wedge |v| \geq 1 \wedge \left( \forall i \in NN \cup \lbrace 0 \rbrace , uv^iw \in L \right)]$

\begin{center}

\titleformat{\section}[block]
  {\hspace{\parindent}}
  {\thesection}
  {1ex}{}

\titlespacing*{\section}{0pt}{*4}{*4}

$\textbf{Задание №1. Построить конечный автомат,
распознающий язык}$\\

\end{center}

$\underline{\text{Автомат №1}}$\\
$L = \lbrace \omega \in \lbrace a,b,c \rbrace ^* | |\omega|_c = 1 \rbrace $\\
\includegraphics{1.1}

$\underline{\text{Автомат №2}}$\\
$L = \lbrace \omega \in \lbrace a,b \rbrace ^* | |\omega|_a \leq 2; |\omega|_b \geq 2 \rbrace $\\

Здесь и далее на вершинах выдаётся число встретившихся "a" и "b"\\
	
\includegraphics{1.2}

$\underline{\text{Автомат №3}}$\\
$L = \lbrace \omega \in \lbrace a,b \rbrace ^* | |\omega|_a \neq |\omega|_b \rbrace $\\

$\text{Проверим является ли данный язык регулярным}\\
\text{Используем лемму о разрастании}\\
\overline L = {\omega \in {a, b}^* | |\omega|_a = |\omega|_b}$\\ 
$\text{Доказательство: }$ \\
$\omega = a^nb^n \in \overline L$\\ 
$|\omega| = 2n \geq n$\\ 
$xy =a^ia^j, \quad i + j \leq n$\\ 
$\omega = a^ia^ja^{n - i - j}b^n$\\ 
$\omega = a^ia^{jk}a^{n - i - j}b^n \notin \overline L \quad ,k > 1 $\\
$\text{Язык не является регулярным, значит и исходный не является.}$ \\ 

$\underline{\text{Автомат №4}}$\\
$L = \lbrace \omega \in \lbrace a,b \rbrace ^* | \omega\omega = \omega\omega\omega \rbrace $\\
$\text{Если }|\omega| > 0, \text{то } \omega\omega\neq\omega\omega\omega,  
\text{значит язык есть пустое слово.}$\\
\includegraphics{1.4}

\titleformat{\section}[block]
  {\hspace{\parindent}}
  {\thesection}
  {1ex}{}

\titlespacing*{\section}{0pt}{*4}{*4}

$\textbf{Задание №2.Построить конечный автомат,используя прямое произведение}$\\

$\underline{\text{Автомат №1}}$\\
$L_1 = \lbrace w \in \lbrace a,b \rbrace ^* ||w|_a \geq 2 \wedge |w|_b \geq 2 \rbrace $\\

$L_1 = {\omega \in {a, b}^* \mid |\omega|_a \geq 2} \cap {\omega \in {a, b}^* \mid |\omega|_b \geq 2}$\\

\includegraphics[scale= 0.8]{2.1}
\includegraphics[scale= 0.8]{2.1.2}

$\Sigma = {a, b} $\\ 
$Q = \lbrace AD, AE, AF, BD, BE, BF, CD, CE, CF \rbrace $\\ 
$S = AD $\\ 
$T = CF $\\ 
$\delta(AD, a) = BD \qquad \delta(BD, a) = CD \qquad \delta(CD, a) = CD$\\
$\delta(AD, b) = AE \qquad \delta(BD, b) = BE \qquad \delta(CD, b) = CE$\\ 
$\delta(AE, a) = BE \qquad \delta(BE, a) = CE \qquad \delta(CE, a) = CE$\\
$\delta(AE, b) = AF \qquad \delta(BE, b) = BF \qquad \delta(CE, b) = CF $\\
$\delta(AF, a) = BF \qquad \delta(BF, a) = CF \qquad \delta(CF, a) = CF $\\
$\delta(AF, b) = AF \qquad \delta(BF, b) = BF \qquad \delta(CF, b) = CF$\\

\includegraphics[scale= 0.8]{2.1.3}


$\underline{\text{Автомат №2}}$\\
$L_2 = \lbrace \omega \in \lbrace a,b \rbrace ^* ||\omega|_a \geq 3 \wedge |\omega|_b \text{нечётное} \rbrace $\\

$L_2 = \lbrace \omega \in \omega \lbrace a, b \rbrace ^* ||\omega| \geq 3 \rbrace \cap \lbrace \omega \in \lbrace a, b \rbrace ^* ||\omega| \text{нечетно} \rbrace$\\

\includegraphics[scale= 0.8]{2.2}\\
\includegraphics[scale= 0.8]{2.2.1}

$\Sigma = \lbrace a, b \rbrace $\\
$Q = \lbrace AE, AF, BE, BF, CE, CF, DE, DF \rbrace $\\
$S = AE$\\
$T = DF$\\
$\delta(AE, a) = BF \qquad \delta(CE, a) = DF$ \\ 
$\delta(AE, b) = BF \qquad \delta(CE, b) = DF$ \\
$\delta(AF, a) = BE \qquad \delta(CF, a) = DE$ \\
$\delta(AF, b) = BE \qquad \delta(CF, b) = DE$ \\
$\delta(BE, a) = CF \qquad \delta(DE, a) = DF$ \\
$\delta(BE, b) = CF \qquad \delta(DE, b) = DF$ \\
$\delta(BF, a) = CE \qquad \delta(DF, a) = DE$ \\
$\delta(BF, b) = CE \qquad \delta(DF, b) = DE$ \\

\includegraphics[scale= 0.8]{2.2.2}

$\underline{\text{Автомат №3}}$\\
$L_3 = \lbrace \omega \in \lbrace a,b \rbrace ^* | |\omega|_a  \text{чётно} \wedge |\omega|_b \text{кратно 3} \rbrace $\\

$L_3 = \lbrace \omega \in \lbrace a, b \rbrace ^* ||\omega|_a \text{четно} \cap \lbrace \omega \in \lbrace a, b \rbrace ^* ||\omega|_b \text{кратно 3} \rbrace $\\

\includegraphics[scale= 0.8]{2.3}
\includegraphics[scale= 0.8]{2.3.1}

$\Sigma = \lbrace a, b \rbrace $\\
$Q = \lbrace C, AD, AE, BC, BD, BE \rbrace $\\
$S = AC $\\*
$T = AC $\\*
$\delta(AC, a) = BC \qquad \delta(BC, a) = AC $\\
$\delta(AC, b) = AD \qquad \delta(BC, b) = BD $\\
$\delta(AD, a) = BD \qquad \delta(BD, a) = AD $\\
$\delta(AD, b) = AE \qquad \delta(BD, b) = BE $\\
$\delta(AE, a) = BE \qquad \delta(BE, a) = AE $\\
$\delta(AE, b) = AC \qquad \delta(BE, b) = BC $\\

\includegraphics[scale= 0.7]{2.3.2}

$\underline{\text{Автомат №4}}$\\
$L_4 = \overline{L_3}$\\

\includegraphics[scale= 0.7]{2.4}

$\underline{\text{Автомат №5}}$\\
$L_5 = L_2 / L_3$\\
$L_5 = L_2 \cap L_4$\\

\includegraphics[scale= 0.8]{2.5}

\begin{center}

\titleformat{\section}[block]
  {\hspace{\parindent}}
  {\thesection}
  {1ex}{}

\titlespacing*{\section}{0pt}{*4}{*4}

$\textbf{Задание №3. Построить минимальный ДКА
по регулярному выражению}$\\

\end{center}

$\underline{\text{Автомат №1}}$\\
$(ab+aba)^* a$\\
\includegraphics[scale = 0.8]{3.1}\\

Построим минимальный НКА по ДКА\\

\begin{table}[h!]
	\centering
    \begin{tabular}{l | l | l |}
    \hline
     & $a$ & $b$ \\ \hline
    $1$ & $3,6,10$ & $\emptyset$ \\ \hline
    $3,6,10$ & $\emptyset$ & $4,7$ \\ \hline
    $4,7$ & $8,3,6,10$ & $\emptyset$ \\ \hline
    $8,3,610$ & $3,6,10$ & $47$ \\
    \hline
    \end{tabular}
\end{table}

\includegraphics[scale= 0.8]{3.1.1}

$\underline{\text{Автомат №2}}$\\
$a(a(ab)^*b)^*(ab)^*$\\

Построим НКА:\\

\includegraphics[scale= 0.8]{3.2}\\

По нему строим ДКА:\\

\begin{table}[h!]
	\centering
    \begin{tabular}{l | l | l |}
    \hline
     & $a$ & $b$ \\ \hline
    $0$ & $17$ & $\emptyset$ \\ \hline
    $17$ & $26$ & $\emptyset$ \\ \hline
    $26$ & $3$ & $57$ \\ \hline
    $3$ & $\emptyset$ & $4$ \\ \hline
    $57$ & $26$ & $\emptyset$ \\ \hline
    $4$ & $3$ & $5$ \\ \hline
    $5$ & $26$ & $\emptyset$ \\ 
    \hline
    \end{tabular}
\end{table}

\includegraphics[scale= 0.8]{3.2.1}

Минимизируем автомат: \\
$\text{1 эквивалентность: }(0,26,3,4,\emptyset),(57,17,5)$\\
$\text{2 эквивалентность: } (0),(26,4),(3,\emptyset),(57,17,5)$\\
$\text{3 эквивалентность: } (0),(26,4),(3),(\emptyset),(57,17,5)$

\includegraphics[scale= 0.8]{3.2.2}

$\underline{\text{Автомат №3}}$\\
$(a + (a + b) (a + b)b)^* $\\

Строим НКА: \\

\includegraphics[scale=0.8]{3.3}\\

Строим ДКА: \\

\begin{table}[h!]
	\centering
    \begin{tabular}{l | l | l |}
    \hline
     & $a$ & $b$ \\ \hline
    $0$ & $01$ & $1$ \\ \hline
    $01$ & $012$ & $12$ \\ \hline
    $1$ & $2$ & $2$ \\ \hline
    $012$ & $012$ & $012$ \\ \hline
    $12$ & $2$ & $02$ \\ \hline
    $2$ & $\emptyset$ & $0$ \\ \hline
    $02$ & $01$ & $01$ \\ 
    \hline
    \end{tabular}
\end{table}

\includegraphics[scale= 0.8]{3.3.1}

Данный автомат будет минимальным\\

$\underline{\text{Автомат №4}}$\\
$(b + c)((ab)^*c + (ba)^* )^* $\\

Построим ДКА: \\

\includegraphics[scale= 0.8]{3.4}

Минимизируем:\\
$\text{1 эквивалентность:} (1,2,3,4,5,\emptyset),(6,7) $\\
$\text{2 эквивалентность:} (1,2,3,\emptyset),(4),(5),(6,7) $\\
$\text{3 эквивалентность:} (1,2,3,\emptyset),(4),(5),(6,7) $\\
$\text{4 эквивалентность:} (1,\emptyset),(2),(3),(4),(5),(6,7) $\\
$\text{5 эквивалентность:} (1),(\emptyset),(2),(3),(4),(5),(6,7) $\\

\includegraphics[scale= 0.8]{3.4.1}\\

$\underline{\text{Автомат №5}}$\\
$(a + b)^+ (aa + bb + abab + baba) (a + b)^+ $ \\

Строим НКА: \\

\includegraphics[scale= 0.8]{3.5}

Преобразуем в минимальный ДКА: \\

\includegraphics[scale= 0.8]{3.5.1}

\begin{center}

\titleformat{\section}[block]
  {\hspace{\parindent}}
  {\thesection}
  {1ex}{}

\titlespacing*{\section}{0pt}{*4}{*4}

$\textbf{Задание №4. Определить является ли язык
регулярным или нет}$\\

\end{center}

$\underline{\text{Автомат №1}}$\\
$L = \lbrace (aab)^nb(aba)^m \mid n \geq 0, m \geq 0 \rbrace $\\

\includegraphics[scale= 0.8]{4.1}

$\underline{\text{Автомат №2}}$\\
$L = \lbrace uaav |u \in \lbrace a, b \rbrace ^*, v \in \lbrace a, b \rbrace ^*, |u|_b \geq |v|_a \rbrace $\\

$\omega = b^naaa^n,|\omega|\geq n $\\
$\omega = xyz $\\
$x=b^i \qquad y=b^j \qquad i+j\leq n \qquad j>0 $\\
$z=b^{n-i-j}aaa^n $\\
$|xy|\leq n \qquad |y|>0 $\\
$xy^0z=b^ib^{n-i-j}aaa^n=b^{n-j}aaa^n \notin L $\\

$\textit{Язык не является регулярным}$\\

$\underline{\text{Автомат №3}}$\\
$L = \lbrace a^mw \mid w \in \lbrace a, b \rbrace ^*, 1 \leq |w|_b \leq m \rbrace $\\

$\omega = a^nb^n , |\omega| \geq n $\\
$\omega = xyz $\\
$x=a^i \qquad y=a^j \qquad i+j \leq n \qquad j>0 $\\
$z=a^{n-i-j}b^n $\\
$|xy| \leq n \qquad |y|>0 $\\
$xy^0z=a^ia^{n-i-j}b^n = a^{n-j}b^n \notin L $\\

$\textit{Язык не является регулярным}$\\

$\underline{\text{Автомат №4}}$\\
$L = \lbrace a^kb^ma^n \mid k = n \vee m > 0 \rbrace $\\

$\omega = a^nba^n , |\omega| \geq n $\\
$\omega = xyz $\\
$x=a^i \qquad y=a^j \qquad i+j \leq n \qquad j>0 $\\
$z=a^{n-i-j}ba^n $\\
$|xy| \leq n \qquad |y|>0 $\\
$xy^kz = a^ia^{jk}a^{n-i-j}ba^n = a^{n-j(k-1)}ba^n \notin L \quad \forall k > 1 $\\

$\textit{Язык не является регулярным}$\\

$\underline{\text{Автомат №5}}$\\
$L = \lbrace ucv \mid u \in \lbrace a, b \rbrace ^*, v \in \lbrace a, b \rbrace ^*, u \neq v^R \rbrace $\\

$\overline{L} = \lbrace ucv \mid u \in \lbrace a, b \rbrace ^*, v \in \lbrace a, b \rbrace ^*, u = v^R \rbrace \\
\text{Используем лемму о разрастании} \\
\omega = b^ncb^n, \omega \in \overline{L}, |\omega| = 2n+1 \\
\text{Для всех } x,y \qquad y=b^i \\
xy^kzk > 1 \\
\text{Язык } \overline{L} \text{ не регулярный} \\
\text{Значит и язык } L \text{ не регулярный.}$\\

\begin{center}

\titleformat{\section}[block]
  {\hspace{\parindent}}
  {\thesection}
  {1ex}{}

\titlespacing*{\section}{0pt}{*4}{*4}

$\textbf{Задание №5. Реализовать алгоритмы}$\\

\end{center}

$1. \text{Построение ДКА по НКА с }\lambda \text{-переходами}$\\

Алгоритм Томпсона строит по НКА эквивалентный ДКА следующим образом:\\
Начало.\\
Шаг 1. Помещаем в очередь Q множество, состоящее только из стартовой вершины.\\
Шаг 2. Затем, пока очередь не пуста выполняем следующие действия:\\
Достаем из очереди множество, назовем его q\\
$\text{Для всех } c \in \Sigma \text{ посмотрим, в какое состояние ведет переход по символу c}\\ \text{из каждого состояния в q.}$\\ 
Полученное множество состояний положим в очередь Q только если оно не лежало там раньше.\\ 
Каждое такое множество в итоговом ДКА будет отдельной вершиной, в которую будут вести переходы по соответствующим символам.\\
Если в множестве q хотя бы одна из вершин была терминальной в НКА, то соответствующая данному множеству вершина в ДКА также будет терминальной.\\
Конец.\\

Пусть дан произвольный НКА:\\
$\left\langle \Sigma, Q, s \in Q, T\subset Q, \delta : Q * \Sigma \rightarrow 2^Q \right\rangle$ \\

Построим по нему следующий ДКА:\\
$\left\langle \Sigma, Q_d, s_d \in Q_d, T_d \subset Q_d, \delta_d : Q_d * \Sigma\rightarrow Q_d \right\rangle $\\


$1. Q_d = \lbrace q_d | q_d \subset 2^Q$\\
$2. s_d = \lbrace s \rbrace $\\
$3. T_d = \lbrace q \in Q_d | \exists p \in T : p \in q \rbrace $\\
$4.\delta_d \left( q,c \right) = \lbrace \delta \left(a,c \right) | a \in q \rbrace $\\

Теорема:\\
Построенный ДКА эквивалентен данному НКА.\\
$\left( \text{Без доказательства} \right) $\\

Пример:\\
Пусть нам дан НКА:\\
\includegraphics{DKA_W}\\

По нашему заданию эквивалентного ДКА мы получаем:\\
\includegraphics{NKA_W}\\

2. Прямое произведение языков, с возможностью построить пересечение,
объединение и разность\\

Прямым произведением двух ДКА:\\
$A1= \langle \Sigma 1, Q1, s1, T1, \delta 1 \rangle \text{и } A2= \langle \Sigma 2, Q2, s2, T2, \delta 2 \rangle \\ \text{Называется ДКА } A= \langle \Sigma,Q ,s ,T ,\delta \rangle, \text{где:}$

%Σ=Σ1∪Σ2 
%Q=Q1×Q2
%s=⟨s1,s2⟩
%T=T1×T2
%δ(⟨q1,q2⟩,c)=⟨δ1(q1,c),δ2(q2,c)⟩

$\Sigma = \Sigma 1 \cup \Sigma 2 \\
Q=Q1 * Q2 \\
s=\langle s1,s2 \rangle \\
T=T1*T2 \\
\delta \left( \langle q1,q2 \rangle ,c2 \right) = \langle \delta 1 \left( q1,c \right) , \delta 2 \left( q2,c \right) \rangle
$\\

Изменив конструкцию, получаем автомат, дающий возможность построить разность или объединение двух языков. \\

\begin{figure}[h]
\centering
\includegraphics{Raz}
\caption{Разность ДКА}
\label{fig:image}
\end{figure}



\begin{figure}[h]
\centering
\includegraphics{Ob}
\caption{Объединение ДКА}
\label{fig:image}
\end{figure}

$\text{Необходимо разрешать любую цепочку,}\\
\text{удовлетворяющую первому или второму автомату.}\\
\text{Делаем терминальными вершины } T= \left( T1*Q2 \right) \cup \left( Q1*T2 \right).\\
\text{Полученный автомат удовлетворяет нашим требованиям,} \\
\text{так как попав в состояние } T1 \text{ или } T2,\text{цепочка будет удовлетворять}\\ \text{первому или второму автомату соответственно.}$\\


\end{document}